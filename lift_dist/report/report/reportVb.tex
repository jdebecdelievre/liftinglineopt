\documentclass[letterpaper,12pt]{article}
\usepackage{tabularx} % extra features for tabular environment
\usepackage{amsmath}  % improve math presentation
\usepackage{graphicx} % takes care of graphic including machinery
\graphicspath{ {images/} }
\usepackage{float}
\usepackage[margin=1in,letterpaper]{geometry} % decreases margins
% \usepackage{cite} % takes care of citations
\usepackage[final]{hyperref} % adds hyper links inside the generated pdf file
\hypersetup{
	colorlinks=true,       % false: boxed links; true: colored links
	linkcolor=blue,        % color of internal links
	citecolor=blue,        % color of links to bibliography
	filecolor=magenta,     % color of file links
	urlcolor=blue         
}

\usepackage{ar}
\usepackage{siunitx}
\usepackage{longtable}
\setlength\LTleft{0pt} 

\begin{document}
\newcommand{\cldat}{\boldsymbol{{c_l}^{2D}}}
\newcommand{\CDp}{C_{D_p}}
\newcommand{\CDi}{C_{D_i}}
\newcommand{\cl}[1]{{c_l}_{\bf #1}}
\newcommand{\cdp}[1]{{c_{d_p}}_{\bf #1}}
\newcommand{\cb}[1]{\left(
	\frac c b 
\right)_{#1}
}
\newcommand{\cddat}{\boldsymbol{{c_{d_p}}^{2D}}}
% \newcommand{\CDi}{C_{D_i}}


\title{Glider Optimization with Lifting Line Theory and Airfoil Wind Tunnel Data}
\author{Jean de Becdelievre}
\date{\today}
\maketitle

\section{Nomenclature}

{\renewcommand\arraystretch{1.0}
\noindent\begin{longtable}{@{}l @{\quad : \quad} l@{}}
$W$  & glider weight \\
$L, C_L$ & lift, lift coefficient \\
$D,Di,D_0, C_D,C_{Di},C_{D_0}$& drag, induced drag, friction drag, and respective coefficients \\
$b$ & wing span \\
$M$ & wing area \\
$\AR$ & wing aspect ratio \\
$y$ & span-wise coordinate, ranging from $-b/2$ to $b/2$ \\
$\theta$ & remapped span-wise coordinate ranging from $0$ to $\pi$ \\
$N_{theta}$ & Number of span-wise sections \\
$\Gamma(y) \text{ or } \Gamma(\theta)$ & span-wise strength of vortex sheet\\
$A_n$ & nth coefficient of the Fourier expansion of $\Gamma$ \\
$N_{A}$ & Number of Fourier coefficients considered for $\Gamma$ \\
$M$ & matrix of size $N_{\theta}, N_A$ such that: $ \forall k,n \quad M_{k,n} = \sin(n\theta_k)$ \\
$M'$ & matrix of size $N_{\theta}, N_A$ such that: $ \forall k,n \quad {M'}_{k,n} = n\sin(n\theta_k)/\sin(\theta_k)$ \\
$A$ & vector of size $N_A$ containing all the $A_n$ \\
$\alpha(y) \text{ or } \alpha(\theta)$ & span-wise angle of attack distribution\\
$\alpha_i(y) \text{ or } \alpha_i(\theta)$ & span-wise lift-induced angle of attack distribution\\
$c(y) \text{ or } c(\theta)$ & span-wise chord distribution\\
$c_l(y) \text{ or } c_l(\theta)$ & span-wise lift coefficient distribution\\
$c_d(y) \text{ or } c_d(\theta)$ & span-wise drag coefficient distribution\\
$V$ & airspeed \\
$\cldat \text{ and } \cddat $ & fit of 2D airfoil data \\
$Re$ & Reynolds number \\
$y_k, \theta_k, \alpha_k, \alpha_{i_k}, c_k, c_{l_k}$ &  $y, \theta, \alpha, \alpha_i, c, c_{l}$ at the kth span section\\
\end{longtable}}


\section{Introduction}

This code aims at optimizing the wing planform and airspeed of a glider of a given weight W and fixed airfoil section for maximum $L/D$.
2D wind tunnel data of the airfoil is available, {\it i.e.} the function
% $\cldat(\alpha, Re)$ and 
$\cddat(c_l, Re)$ are provided for  $c_l \in [\cl{lb}, \cl{ub}]$   and $Re \in [{Re}_{\bf lb}, {Re}_{\bf ub}]$.

Schematically, the optimization problem reads:
\begin{align*}
	& \underset{V, b, c(y), c_l(y)}{\text{minimize}} &  C_D &/ C_L  & \\
	& \text{subject to} 
	&  {C_L}^2 + {C_D}^2 &= {C_W}^2 \quad \text{-} \quad \text{\it gliding flight constraint} \\
	& & \cl{lb} \leq &\cl \leq \cl{ub} \quad \text{-} \quad \text{\it domain constraint for } \cddat  \\
	& & {Re}_{\bf lb} \leq &{Re} \leq {Re}_{\bf ub} \quad \text{-} \quad \text{\it domain constraint for } \cddat
\end{align*}

First, the optimization problem is described in more details. 
Second, some insights are obtained from considering simple functions for $\cddat$.
Finally, using a neural network fit of 2d airfoil data, the complete problem is solved and results are analysed.


\section{Optimization Problem}

\paragraph{Loss Function:}Using the gliding flight constraint, we have:
$$ \frac{{C_L}^2}{{C_D}^2} = \frac{{C_W}^2 - {C_D}^2}{{C_D}^2} = \frac{{C_W}^2}{{C_D}^2} - 1$$
Therefore, minimizing $C_{D} / C_W$  is equivalent to minimizing $C_{D} / C_L$.
Moreover, $C_D$ can be decomposed into lift-induced drag and parasite drag:
$$C_D = \CDp + \CDi$$

\paragraph{Remapping the spanwise coordinate:} The spanwise coordinate $y \in [\frac {-b} 2, \frac b 2]$ can be remapped into $\theta \in [\pi, 0]$, such that:
$$y = \frac b 2 \cos\theta$$

\paragraph{Parasite Drag :}Since we can compute the 2D parasite drag using $\cddat$, the total $\CDp$ is obtained with integration.
\begin{align*}
	\CDp &= \frac 1 S \int_{-b/2}^{b/2} c(y) \cddat(c_l(y), Re(y)) dy\\
	&= \frac b {2S} \int_{0}^{\pi} c(\theta) \cddat(c_l(\theta), Re(\theta)) \sin(\theta) d\theta\\
	&= \frac b {2S} \sum_{k=0}^{N_y} w_k c_k \cdp{k}
\end{align*}
where $N_y$ is the number of spanwise panels, $\theta_{1,\dots, N_y}$ their location, 
and $w_{1,\dots, N_y}$ a set of quadrature weights that depend on the spacing and location of the panels. We also wrote:
$$c(\theta_k) = c_k \quad \text{and} \quad \cddat(c_l(\theta_k), Re(\theta)k)) = \cdp{k}$$
In the rest of this document, we almost always describe the chord distribution with the non-dimensional elements $c_k/b$, that we write $\cb{k}$. In the expression of $\CDp$, this gives:
$$\CDp = \frac {\AR} {2} \sum_{k=0}^{N_y} w_k \cb{k} \cdp{k}$$

\paragraph{Induced Drag:}We use a far field estimate of the induced drag. 
Assuming a planar wake, we use a sine basis expansion of the vortex sheet strength:
$$\Gamma(\theta) = 2bV \sum_{n=1}^{N_a} A_n sin(n \theta)$$
$N_a$ is the order of this expansion.
This gives us:
$$\CDi = \pi \AR \sum_{n=1}^{N_a} n A_n^2$$

\paragraph{Weight Coefficient:}Factorizing by $\AR$, we write the weight coefficient as: 
$$C_W = \frac{W}{1/2 \rho V^2 S} = \frac{W}{1/2 \rho }\AR (Vb)^{-2} $$
In this study, $\rho$ is assumed to be constant, and will be dropped from the objective function below.

\paragraph{Objective Function:}Putting together all of the above, we obtain the following objective function:
$$\frac{C_D}{C_W} = (Vb)^2\left( \pi \sum_{n=1}^{N_a} n A_n^2 +   \frac {1} {2} \sum_{k=0}^{N_y} w_k \cb{k} \cdp{k} \right)$$

\paragraph{Approximate Handling of the Equality Constraint:} We simplify the problem by assuming that the glide angle $\gamma$ is small.
In such case,
\begin{align}
	C_W = \frac{C_L}{\cos \gamma} \approx C_L = \pi \AR A_1
	\label{eq:A1}
\end{align}
The validity of the small angle assumption is discussed in appendix \ref{apx:smallangle}.
Instead of solving the original nonlinear constrained problem, we insert the value of $A_1$
 into the objective function. $\pi \AR A_1^2$ appears in the expression of $\CDi$, so we get: 
 \begin{align*}
	\CDi &= \pi \AR A_1^2 + \pi \AR \sum_{n=2}^{N_a} n A_n^2\\
		&=  \frac{\AR}{\pi} \left(\frac{C_W}{\AR} \right)^2 + \pi \AR \sum_{n=2}^{N_a} n A_n^2\\
		&=  \frac{4W^2}{\rho^2 \pi} \AR (Vb)^{-4} + \pi \AR \sum_{n=2}^{N_a} n A_n^2
 \end{align*}

\paragraph{Inequality Constraints:} We can write the Reynold's number and the lift coefficient $\cl{k}$ at the local section k as a function of other variables used previously:
$$
	\cl{k} = 4 \cb{k}^{-1} \sum_{n=1}^{N_a} A_n \sin(n \theta_k)
	 = 4 \cb{k}^{-1} \left(A_1 \sin(\theta_k) + \sum_{n=2}^{N_a} A_n \sin(n \theta_k) \right)
$$

\noindent Therefore:
$$\cl{k} = 4 \cb{k}^{-1} \left(\frac{W}{1/2 \rho \pi} (Vb)^{-2} \sin(\theta_k) + \sum_{n=2}^{N_a} A_n \sin(n \theta_k) \right) $$

\noindent Also:
$$Re_{k} = \frac{\rho}{\nu}\cb{k}\sqrt{(Vb)^2} $$


The collected optimization problem is written and analyzed in the next section.

\section{Optimization Problem Insights and Discussion}

Putting together all of the last section we have:

\begin{align*}
	& \underset{(Vb)^2, \cb{1:N_y}, A_{2:N_a}}{\text{minimize}}  &
	\frac{4W^2}{\rho^2 \pi} (Vb)^{-2} + &
	(Vb)^2\left( 
	\pi \sum_{n=2}^{N_a} n A_n^2 +  
	\frac {1} {2} \sum_{k=0}^{N_y} w_k \cb{k} \cddat(\cl{k}, Re_k) \right) & \\
	& \text{subject to} & \cl{lb} \leq &\cl \leq \cl{ub} \\
	& & {Re}_{\bf lb} \leq &{Re} \leq {Re}_{\bf ub}
\end{align*}
with:
\begin{align*}
	& \cl{k} = 4 \cb{k}^{-1} \left(\frac{W}{1/2 \rho \pi} (Vb)^{-2} \sin(\theta_k) + \sum_{n=2}^{N_a} A_n \sin(n \theta_k) \right) \\
	& Re_{k} = \frac{\rho}{\nu}\cb{k}\sqrt{(Vb)^2} 
\end{align*}


% \begin{align*}
% 	& \underset{(Vb)^2, \cb{1:N_y}, A_{2:N_a}}{\text{minimize}}  &
% 	\frac{4W^2}{\rho^2 \pi} (Vb)^{-2} + &
% 	(Vb)^2\left( 
% 	\pi \sum_{n=2}^{N_a} n A_n^2 +  
% 	\frac {1} {2} \sum_{k=0}^{N_y} w_k \cb{k} \cdp{k} \right) & \\
% 	& \text{subject to} & \cl{lb}/4 \leq & \cb{k}^{-1} \sum_{n=1}^{N_y} A_n \sin(n \theta_k) \leq \cl{ub}/4 \\
% 	& & \frac{\nu}{\rho} {Re}_{\bf lb} \leq & \cb{k}\sqrt{(Vb)^2}  \leq \frac{\nu}{\rho} {Re}_{\bf ub}
% \end{align*}
% with:
% \begin{align*}
% 	& \cdp{k} = \cddat(\cl{k}, Re_k) \\
% 	& \cl{k} = 4 \cb{k}^{-1} \sum_{n=0}^{N_y} A_n \sin(n \theta_k) \\
% 	& Re_{k} = \frac{\rho}{\nu}\cb{k}\sqrt{(Vb)^2} 
% \end{align*}

\paragraph{Design Variables:} The first interesting note is that the optimization of the planform and airspeed only requires:
\begin{itemize}
	\item $(Vb)^2$ which is the only place where the airspeed appears. Note that we could equivalently use $C_W / \AR$, which makes it obvious that this design variable is tightly connected to the nominal $C_L$.
	\item $\cb{1:N_y}$ is simply related to the planform shape
	\item $A_{1:N_a}$ describes the lift distribution. Note that $A_1$ is not a design variable, it is directly set by the value of $C_W / \AR$ (see equation \ref{eq:A1}).
\end{itemize}
Once the optimization has converged, choosing either a span value $b$, an airspeed $V$, or the chord $c_k$ of any of the chord sections allows to recover the full dimensional planform shape.

\paragraph{Origin of each term:}In the objective function, we can connect all of the terms with their physical origin:
$$
\underbrace{\frac{4W^2}{\rho^2 \pi} (Vb)^{-2}}_{(1)} + 
\underbrace{(Vb)^2}_{\text{division by $C_W$}}
\left( 
\underbrace{\pi \sum_{n=2}^{N_a} n A_n^2}_{(2)} +  
\underbrace{
\frac {1} {2} \sum_{k=0}^{N_y} w_k \cb{k} \cddat(\cl{k}, Re_k)}_{\text{parasite drag}} \right)$$
$(1)$ and $(2)$ are the two parts of the induced drag, divided by $C_W$:
\begin{itemize}
\item $(1)$ comes from the induced drag of an elliptically loaded wing. In essence, it is proportional to $C_L / (\pi \AR)$.
\item $(2)$ is the sum of all additional terms due to the non elliptical loading.
\end{itemize}

The rest of this section develops some insights for this problem.

\section{Insights For Individual Design Variable}

\subsection{$(Vb)^2$}

Fixing all other design variables, the optimization problem from $(Vb)^2$ looks like the following:
\begin{align*}
	& \underset{(Vb)^2}{\text{minimize}}  &
	\frac{k_1}{(Vb)^{2}} &+ (Vb)^2 k_2  \\
	& \text{subject to} & lb_1 \leq & 1 / (Vb)^2 + k_3 \leq ub_1 \\
	& & lb_2 \leq & \sqrt{(Vb)^2} \leq ub_2
\end{align*}
where $k_1$, $k_2$ and $k_3$ are constant positive numbers.
%
The landscape for this function is shown on figure \ref{fig:Vb2img}

\begin{figure}[h]
	\centering
	\caption{Optimization landscape for $(Vb)^2$}
	\label{fig:Vb2img}
\end{figure}

\subsection{$\cb{1:N_y}$}

Fixing all other design variables, the optimization problem from $\cb{1:N_y}$ looks like the following:
\begin{align*}
	& \underset{\cb{1:N_y}}{\text{minimize}}  &
	\sum_{k=1}^{N_y} &k_i \cb{k} \cddat{\cb{k}}  \\
	& \text{subject to} & lb \leq & \cb{k} \leq ub
\end{align*}
where $k_{0:N_y}$ are constant positive numbers, 
and ($lb$, $ub$) are bounds derived from the $c_l$ and $Re$ bounds.

Very clearly, unless $\cddat$ varies with $\cb{k}^{-1}$ to a power bigger than one,
the optimizer will always set each $\cb{k}$ to its lowest bound 
(from the $Re$ or from $c_l$, whichever is tightest).
Interestingly, if the airfoil is a flat plate, then:
$$\cddat(Re) \approx 2 \frac{0.074}{Re^{0.2}}$$
With only a power 0.2 on $\cb{k}^{-1}$, this is not a big enough penalty on small chords to prevent the maximum L/D to happen at the lowest possible chord.

\subsection{$A_{1:N_a}$}

Fixing all other design variables, the optimization problem from $A_{2:N_a}$ looks like the following:
\begin{align*}
	& \underset{A_{2:N_a}}{\text{minimize}}  &
	 \sum_{n=2}^{N_a} nA_n^2   &+  \sum_{k=0}^{N_y} k_i \cddat(\sum_{n=2}^{N_a} A_n sin(n \theta_k))\\
	& \text{subject to} & lb_k \leq & \sum_{n=2}^{N_a} A_n sin(n \theta_k) \leq ub_k
\end{align*}
where ($lb_k \leq 0$) and ($ub_k \geq 0$) are bounds derived from the $c_l$ bounds.
Pretty clearly, the main incentive is to set all of the 

\section{Confirming the Insight on Simplified 2D Sections}
\subsection{Constant $\cddat$}

Let's assume:
$$\cddat(\cl, Re) = \cdp \quad - \quad \text{\it (constant)}$$
Then the optimization problem becomes:

\begin{align*}
	& \underset{(Vb)^2, \cb{1:N_y}, A_{2:N_a}}{\text{minimize}}  &
	\frac{4W^2}{\rho^2 \pi} (Vb)^{-2} + &
	(Vb)^2\left( 
	\pi \sum_{n=2}^{N_a} n A_n^2 +  
	\frac {\AR} {2} \right) & \\
	& \text{subject to} & \cl{lb} \leq &\cl \leq \cl{ub} \\
	& & {Re}_{\bf lb} \leq &{Re} \leq {Re}_{\bf ub}
\end{align*}
with:
\begin{align*}
	 \cl{k} &= 4 \cb{k}^{-1} \sum_{n=1}^{N_y} A_n \sin(n \theta_k)   \\
	 &= 4 \cb{k}^{-1}\left(\sum_{n=1}^{N_y} A_n \sin(n \theta_k) \sum_{n=2}^{N_y} A_n \sin(n \theta_k) \right)\\
	 Re_{k} &= \frac{\rho}{\nu}\cb{k}\sqrt{(Vb)^2} 
\end{align*}


\subsection{Flat Plate Model}
\subsection{Flat Plate with Exponential $C_L$ Penalty}

\appendix

\section{Validity of the small angle assumptions}
\label{apx:smallangle}
TBD

\end{document}


